\documentclass[12pt]{ctexart}
%\setmainfont{標楷體}
%\usepackage{ctex}

\usepackage{xcolor}
\usepackage{amsmath} %數學庫
\usepackage{setspace}
\usepackage{fancyhdr} %排版用,頁數
\usepackage[colorlinks=true, linkcolor=black, citecolor=black, urlcolor=blue]{hyperref}

\usepackage{graphicx} %插入图片的宏包
\usepackage{float} %设置图片浮动位置的宏包
\usepackage{subfigure} %插入多图时用子图显示的宏包
\usepackage{chngpage}


%--------------------color---------------------------%
\definecolor{group_color_1}{HTML}{1D4765}
\definecolor{group_color_2}{HTML}{E5e6e1}
\definecolor{group_color_3}{HTML}{A7d1d9}
\definecolor{group_color_4}{HTML}{bebbb7}

%--------------------color---------------------------%


%畫流程圖
\usepackage{tikz, mathpazo}
\usetikzlibrary{shapes.geometric, arrows}


\renewcommand{\baselinestretch}{1.5}
\renewcommand{\contentsname}{目錄}

\title{ 高一專題統整 }
\author{ 郭勝威 }
\date{ \today }

\begin{document}

%----------------初始化------------------%

%--------流程圖樣式---------------%
\tikzstyle{startstop} = [rectangle, rounded corners, minimum width=3cm, minimum height=1cm,text centered,text=white, draw=black, fill=group_color_1]
\tikzstyle{io} = [trapezium, trapezium left angle=70, trapezium right angle=110, minimum width=3cm, minimum height=1cm, text centered, draw=black, fill=group_color_2]
\tikzstyle{process} = [rectangle, minimum width=3cm, minimum height=1cm, text centered, draw=black, fill=group_color_3]
\tikzstyle{decision} = [diamond, minimum width=3cm, minimum height=1cm, text centered, draw=black, fill=group_color_4]
\tikzstyle{arrow} = [thick,->,>=stealth]
%--------流程圖樣式---------------%


%---------縮放------------%
\tikzset{global scale/.style={
    scale=#1,
    every node/.append style={scale=#1}
  }
}
%---------縮放------------%

%---------------初始化-------------------%
\pagestyle{empty}
\maketitle

\thispagestyle{empty}


\clearpage
\tableofcontents

\pagestyle{plain}
\setcounter{page}{0}

\clearpage
\section{ 序言 }

高一轉瞬即逝, 
當我回顧一年來的收穫,
不知不覺中就做了許多小專案,
每一個專案都是心血來潮才開始的,
實作時,
總會遇到許多問題,
不過我會投入許多的時間去思考不同的解決方案,
抑或從網路上尋找解答,
我相信這一點一滴了累積,
即使不起眼,
但仍代表我的進步

\hspace*{\fill}\\

我把這一年所做的專案分成六個部分,
大部分都是由\textit{python}實作的,
網頁則是用\textit{html, css, javascript}實作,
\hyperref[sec::crawler]{爬蟲}的大部分都是為了讓自己可以輕鬆一點,
其他的地方則是可以因為自己的好奇心而去研究的,
每一個專案都至少花了\textbf{10hr} 以上,
不過卻都沒有很完整,
都還有許多空間可以改進,
不過許多都因為自己的能力未到而尚須時間,
就先把這些託付給未來的我了。



\clearpage
\section{ 爬蟲 }
\label{sec::crawler}

\subsection{實作流程} 
實作流程都相差無幾,可以使用下面這張圖來代表
\hspace*{\fill}\\


%-------------------------流程圖實作-----------------------------%
\begin{figure}[H]
\centering
\begin{tikzpicture}[node distance=2cm][H]
 %定义流程图具体形状
\node (start) [startstop] {觀察網頁};
\node (in1) [process, below of=start] {分析流程};
\node (pro1) [process, below of=in1] { 開始寫程式 };
\node (dec1) [decision, below of=pro1, yshift=-0.5cm] { 遇到問題};
\node (dec2) [decision, below of=dec1, yshift=-1cm] { 進行測試};
\node (pro2b) [process, right of=dec1, xshift=2cm] { 學習新知識};
\node (stop) [startstop, below of=dec2, yshift=-0.5cm] {成果};

 %连接具体形状
\draw [arrow](start) -- (in1);
\draw [arrow](in1) -- (pro1);
\draw [arrow](pro1) -- (dec1);
\draw [arrow](dec1) -- (dec2);
\draw [arrow](dec1) -- (pro2b);
\draw [arrow](dec1) -- node[anchor=west] {No} (dec2);
\draw [arrow](dec1) -- node[anchor=south] {Yes} (pro2b);
\draw [arrow](pro2b) |- (pro1);
\draw [arrow] (stop) to [bend left=60] (start) node[anchor=east, yshift=-4.7cm, xshift=-3cm]{改進};
\draw [arrow](dec2) -| node[ anchor=south, xshift=-1.75cm ]{fail}(pro2b);
\draw [arrow](dec2) -- node[ anchor=west ]{Access}(stop);

\end{tikzpicture}
\end{figure}
%-------------------------流程圖實作-----------------------------%



\subsection{ 抓取解題網站的資料}
\paragraph{ 實作動機 }
\begin{adjustwidth}{2em}{0em}
    因為我有擔任程式設計的教學,
    所以常常需要出題,
    而解題網站 \href{https://cses.fi/}{CSES} 有許多很好的題目,
    所以我有時會把裡面的題目翻譯然後搬到 \href{https://dandanjudge.fdhs.tyc.edu.tw/}{DDJ}上,
    但是\textbf{CSES}裡面的測資是.txt的,
    還要一個一個下載下來,
    而\textbf{DDJ}要求的是.in, .out檔,
    所以我決定寫個程式把檔案下載下來然後改成.in, .out檔
\end{adjustwidth}

\paragraph{ 遇到的困難和解決方式 }
\begin{adjustwidth}{2em}{0em}
    一開始的時候,
    因為才剛開始寫\textit{python},
    所以程式碼十分冗長,
    我自己也看不太懂,
    所以之後就開始嘗試把許多東西變成函式,
    意外的發現挺方便的。
\end{adjustwidth}

\paragraph{學到的事情}
\begin{adjustwidth}{2em}{0em}
\begin{enumerate}
    \item 開始嘗試用\textit{python}拼出想要的功能
    \item 學會使用\textit{urllib}庫,
    \item 學會使用\textit{with...open...}來開啟和寫入檔案
    \item 如何觀察\textit{html}文檔
\end{enumerate}
\end{adjustwidth}

\clearpage
\paragraph{過程}
\hspace*{\fill}\\

%---------------2.2流程圖------------------------%

\begin{figure}[H]
\centering
\begin{tikzpicture}[node distance=3cm][H]

\node (start) [startstop]  { 觀察CSES };
\node (pro1) [process, below of=start, yshift=-1.5cm] { 寫程式 };
\node (problem_2) [ process, below of=pro1, yshift=-0.5cm ] {學習\textit{urllib}庫};
\node (problem_1) [ process, left of=problem_2, xshift=-1.5cm ] {學習\textit{html}庫};
\node (problem_3) [ process, right of=problem_2, xshift=1.5cm ] {學習函式編程};
\node (test) [ process, below of=problem_2, yshift=-0.5cm ] {成果和測試};
\node (stop) [ startstop, below of=test, yshift=-1.5cm] {一個段落};

\draw [arrow](start) -- (pro1);
\draw [arrow](pro1) -| (problem_1);
\draw [arrow](pro1) -- (problem_2);
\draw [arrow](pro1) -| node[anchor=east, yshift=-1.5cm]{程式碼冗長}  (problem_3);
\draw [arrow](problem_1) -- (test);
\draw [arrow](problem_2) -- (test);
\draw [arrow](problem_3) -- (test);
\draw [arrow](test) -- (stop);
\draw [arrow, dash pattern=on 20pt off 20pt, dash phase=10pt, ultra thick](stop) to [out=0, in=0, xshift=2] node[anchor=west]{新的想法?}(start);

\end{tikzpicture}
\end{figure}

%---------------2.2流程圖------------------------%

\clearpage
\paragraph{改進空間}
\begin{enumerate}
    \item 
        因為我實作登陸是用\textit{cookie},
        每一次都需要複製網站上的\textit{cookie},
        所以可以使用\textit{session}之類的庫把他改成輸入帳號跟密碼就可以登錄。
    \item 
        通用的改進: 我想要把他弄出圖形化介面
\end{enumerate}

\paragraph{心得}
\begin{adjustwidth}{2em}{0em}
    這一次實作的功能十分簡單,
    只有使用\textit{python}爬取檔案,
    然後寫入而已。
    不過仍花了我許多時間,
\end{adjustwidth}


\clearpage

\subsection{ 自動機器人 }

%\paragraph{ 實作動機 }
%\paragraph{ 遇到的困難和解決方式 }
%\paragraph{ 學到的事情 }
%t\paragraph{ 過程 }

\paragraph{ 實作動機 }
\begin{adjustwidth}{2em}{0em}
    之前因為疫情所以開始了線上上課,
    但是每一次上課的時候都須找尋找 google meet 的連結,
    十分的不方便,
    因此我就萌生了一個想法:
    可不可以在每節下課的時候發出下一節上課的meet,
    讓他自動把連結送到聊天室,
    這樣除了我自己可以方便的進教室外,
    其他同學也一樣可以方便的使用。
\end{adjustwidth}

\paragraph{遇到的困難}
\begin{adjustwidth}{2em}{0em}

message api:
\begin{adjustwidth}{2em}{0em}
\begin{enumerate}
    \item 
        實作message api的時候發現要伺服器 $\implies$ 使用\href{https://www.heroku.com/}{\textbf{heroku}}
    \item 
        需要一個群組才能開始傳送訊息 $\implies$ 這時候我決定換一個做法
\end{enumerate}
\end{adjustwidth}


\hspace*{\fill}\\
selenium庫: 

\begin{adjustwidth}{2em}{0em}
\begin{enumerate}
    \item
        網頁十分複雜,不知道哪一個地方代表按鈕$\implies$ 研究了很久之後發現他們是直接用div來表示按鈕
\end{enumerate}
\end{adjustwidth}

\end{adjustwidth}

\paragraph{ 學到的事情 }
\begin{adjustwidth}{2em}{0em}
\begin{enumerate}
    \item 
        學會了如何分析更複雜的網頁
    \item
        學會了使用\textit{selenium}庫
\end{enumerate}

    
\end{adjustwidth}

\clearpage
\paragraph{ 過程 }
\hspace*{\fill}\\

%----------------2.3流程圖-------------------%

\begin{figure}[H]
\centering
\begin{tikzpicture}[node distance=3cm, global scale=0.7][H]

\node (start) [startstop] {想法萌出};
\node (find_method) [process, below of=start] {探索可能的方法};
\node (message_api) [process, below of=find_method,xshift=-5cm] {message api};
\node (python_selenium) [process, right of=message_api, xshift=7cm] {python selenium};

%---message_api problem---%
\node (need_server)  [align=center, process, below of=message_api, xshift=-2.5cm] {需要一個\\伺服器};
\node (need_message_group) [process, align=center, right of=need_server, xshift=2cm] {需要一\\個群組};
%---message_api problem---%

%---python selenium problem---%
\node (complex_web) [align=center, process, below of=python_selenium] {網頁\\複雜};
%---python selenium problem---%
    
%-----the thing I get-------%
\node (give_up_message_api) [align=center, startstop, below of=message_api, yshift=-3cm] {把想法擱置};
\node (start_learning_selenium) [align=center, process, below of=complex_web] {開始學習\\selenium};
%-----the thing I get-------%

%-------end-------%
\node (selenium_test) [align=center, process, below of=start_learning_selenium] {測試};
\node (have_end) [align=center, startstop, below of=selenium_test] {成果};
%-------end-------%

%-------draw the flowchart-------%
\draw [arrow] (start) -- (find_method);
\draw [arrow] (find_method) -- (message_api);
\draw [arrow] (find_method) -- (python_selenium);

\draw [arrow] (message_api) -- (need_server);
\draw [arrow] (message_api) -- (need_message_group);
\draw [arrow] (need_server) -- (give_up_message_api);
\draw [arrow] (need_message_group) -- (give_up_message_api);

\draw [arrow] (python_selenium) -- (complex_web);
\draw [arrow] (complex_web) -- node[anchor=west, align=center]{花了許多\\時間理解} (start_learning_selenium) ;

\draw [arrow] (start_learning_selenium) -- (selenium_test);
\draw [arrow] (selenium_test) -- (have_end);
%-------draw the flowchart-------%


\end{tikzpicture}
\end{figure}

%----------------2.3流程圖-------------------%

\clearpage
\subsection{心得}
這個專案是我第一次研究大型公司的網頁,
我發現他們的網頁都十分的複雜,
無論是元素的命名,
還是他作用的方式,
都大幅的讓我了解到自己的不足,
對網頁的了解還只是皮毛,
更加深了我對網頁的興趣。

\subsection{改進}
\begin{adjustwidth}{2em}{0em}
\begin{enumerate}
    \item
        現在的程式仍需要每天清自去點開來執行,
        有些許的不方便
    \item 
        在修改課程的時候有寫麻煩,
        在使用的最後幾天因為調課頻繁的原因出了許多差錯,
        應該可以在多做一個程式來專門修改課程
    \item  
        圖形化介面
\end{enumerate}
\end{adjustwidth}



\clearpage
\subsection{ 密碼破解 }
\paragraph{ 實作動機 }
\begin{adjustwidth}{2em}{0em}
    在學了許多的知識之後,
    有時候總會冒出一些奇妙的想法,
    在學校的時後,
    我發現有些幹部他們登陸時候的密碼十分的簡單,
    許多都只有四位數,
    於是我就萌生了一個想法,
    是不是可以透過程式來破解密碼,
    稍微計算了一下,
    如果每秒鐘請求十次的話,
    對於四位數的密碼共需要請求約1000秒,
    也就是16分鐘,
    是在可以接受的範圍,
    於是我就開始打程式了
\end{adjustwidth}

\paragraph{ 遇到的困難和解決方式 }
\begin{adjustwidth}{2em}{0em}
    這次因為只是簡單的嘗試所以沒有遇到什麼困難,
    不過為了確保不會讓學校網站當掉所以有壓低了請求速率。
\end{adjustwidth}
\paragraph{ 學到的事情 }
\begin{adjustwidth}{2em}{0em}
    在實作過後我學到了\textbf{密碼一定要好好設},
    如果只有數字而且只有四位數的話,
    雖然看起來很難被猜到,
    不過對於電腦來說只是幾分鐘的事情,
    同時也因為了這次嘗試,
    我開始對網路安全有了些許興趣,
    也有開始去學習一些漏洞向觀的知識(譬如: xss, sql注入..., ddos)
\end{adjustwidth}

\clearpage

\paragraph{ 過程 }

\hspace*{\fill}\\
%--------flow chart 2.5------------%
\begin{figure}[H]
\centering
\begin{tikzpicture}[node distance=3cm][H]

\node (start) [startstop] {觀察};
\node (caculate) [process, below of=start] {計算流量};
\node (coding) [process, below of=caculate] {打程式};
\node (end) [process, below of=coding] {測試};
\node (final) [startstop, below of=end] {成品};
\node (think2) [startstop, align=center, below of=final] {了解到了設定\\密碼的重要性};
\node (think1) [startstop, align=center, left of=think2, xshift=-3cm] {對網路安全\\有了興趣};
\node (think3) [startstop, align=center, right of=think2, xshift=3cm] {學習到了\\許多漏洞};

\draw [arrow](start) -- (caculate);
\draw [arrow](caculate) -- (coding);
\draw [arrow](coding) -- (end);
\draw [arrow](end) -- (final);
\draw [arrow, dash pattern=on 2pt off 3pt on 3pt off 2pt](final) -- (think2);
\draw [arrow, dash pattern=on 2pt off 3pt on 3pt off 2pt](final) -- (think1);
\draw [arrow, dash pattern=on 2pt off 3pt on 3pt off 2pt](final) -- (think3);


\end{tikzpicture}
\end{figure}
%--------flow chart 2.5------------%


\clearpage
\subsection{ 校內廣播 }
\clearpage
\subsection{ 排名爬取 }



\clearpage
\section{ 遊戲}
\label{sec::game}
\subsection{ 貪吃蛇 (python) }
\clearpage
\subsection{ 貪吃蛇 (c++) }

\clearpage
\section{ 網頁製作 }
\label{ sec::website }
\subsection{ 乘法練習網站 }
\clearpage
\subsection{ 決策器製作 }
\clearpage
\subsection{ 數學公式製作 }

\clearpage
\section{ 應用程式製作 }
\label{ sec::app }
\subsection{ 搜尋單字 (半完成) }


\clearpage
\section{ manim 動畫製作 }
\label{ sec::animate }
\subsection{ 餘弦證明動畫 }
\clearpage
\subsection{ 圓周弦與弧的性質 }



\clearpage
\section{ 人工智慧 }
\label{ sec::AI }
\subsection{ MINST手寫辨識 }
\clearpage
\subsection{ 個人思考 }

\end{document}